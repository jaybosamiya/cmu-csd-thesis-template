% -*- mode: latex; TeX-master: "./thesis" -*-

% This file is Jay Bosamiya's common preamble code that is useful across
% multiple papers. If something is specific to the current paper, it
% should go into `00-paper-commands.tex` instead.

% Flag for distinguishing submission commands
\newif\ifsubmit{}
\submittrue{}
\submitfalse{}                  % Comment this line out to get the submission view

% Add useful commands for commenting things, adding todos,
% etc. Comments are completely ignored in submission mode, but todos
% and citationneeded commands will show warnings in submission
% mode. This is to ensure that we don't miss out on any important
% todos. At the time of submission, we should hopefully have 0
% warnings overall.
\ifsubmit%
  \newcommand*{\comment}[1]{}
  \newcommand*{\citationneeded}[1]{\PackageWarning{SubmissionCheck}{Citation still needed: #1}}
  \newcommand*{\todo}[1]{\PackageWarning{SubmissionCheck}{TODO still remains: #1}}
  \newcommand*{\maxpageshere}[1]{%
    \ifnum\value{page}>#1%
      \PackageWarning{SubmissionCheck}{Page count is \thepage; max allowed: #1}%
    \fi%
  }
  \newcommand*{\word}[0]{\PackageWarning{SubmissionCheck}{Missing word on page \thepage}\textcolor{red}{WORD}\xspace}
  \newcommand*{\maybecut}[1]{\PackageWarning{SubmissionCheck}{Expected to cut text on \thepage}\textcolor{red}{\sout{#1}}}
  \newcommand*{\xxx}[0]{\PackageWarning{SubmissionCheck}{XXX on page \thepage}\textcolor{red}{XXX}\xspace}
\else
  \renewcommand*{\linenumberfont}{\normalfont\tiny\sffamily\color{lightgray}}
  \linenumbers{}
  \newcommand*{\comment}[1]{\ding{43}\textcolor{Plum}{[#1]}}
  \newcommand*{\citationneeded}[1]{\ding{46}\textcolor{purple}{[cite:#1]}}
  \newcommand*{\todo}[1]{\ding{45}\textcolor{BrickRed}{[TODO\@: #1]}}
  \newcommand*{\maxpageshere}[1]{\par{}\textcolor{BrickRed}{\hrule\hfill[Page limit: #1]\hfill~\hrule}\par{}}
  \newcommand*{\word}[0]{\textcolor{BrickRed}{[WORD]}\xspace}
  \newcommand*{\xxx}[0]{\par\noindent\textcolor{RubineRed}{\rule{\textwidth}{1pt}}\\\centerline{\textcolor{RubineRed}{[XXX]}}\\\textcolor{RubineRed}{\rule{\textwidth}{1pt}}}
  \newcommand*{\maybecut}[1]{\textcolor{Dandelion}{[Consider cutting: \sout{#1}]}}
\fi

\newcommand*{\submissionreminder}[1]{\PackageWarning{SubmissionCheck}{#1}}

% Set up cref to capitalize things
\crefname{chapter}{Chapter}{Chapters}
\Crefname{chapter}{Chapter}{Chapters}
\crefname{section}{Section}{Sections}
\Crefname{section}{Section}{Sections}
\crefname{figure}{Figure}{Figures}
\Crefname{figure}{Figure}{Figures}
\crefname{table}{Table}{Tables}
\Crefname{table}{Table}{Tables}
\crefname{appendix}{Appendix}{Appendices}
\Crefname{appendix}{Appendix}{Appendices}

% Set up the appendices environment to correctly report the name to cref
\xapptocmd\appendices{%
  \crefalias{section}{appendix}%
  \crefalias{subsection}{appendix}%
  \crefalias{subsubsection}{appendix}%
}{}{\PatchFailed}

% Add an `inlineenumerate` environment for inline lists
\newlist{inlineenumerate}{enumerate*}{1}
\setlist[inlineenumerate]{label=(\roman*)}   % chktex 36 (suppress chktex warning about wanting a space before the parens)

% Make sure we're placing no unnecessary line separation in lists like
% enumerate/itemize
\setlist{nosep}

% Prevent LaTeX from complaining about `\\` in section titles and
% such, which are used for providing the PDF bookmarks (that don't
% support newlines)
\pdfstringdefDisableCommands{%
\def\\{}% Newline should become nothing
\def\,{ }% A short space should become a real space
\def\!{}% A negative space should become nothing
}

% Better defaults for tabularx. Set vertical centering on all `X` type
% columns, and create a new column type `Y` that does horizontal
% centering on top of `X`, and a new column type `Z` that does
% right-alignment on top of `X`.
\def\tabularxcolumn#1{m{#1}}
\newcolumntype{Y}{>{\centering\arraybackslash}X}
\newcolumntype{Z}{>{\hfill\arraybackslash}X}

% Clear out a warning thrown by microtype
\microtypecontext{spacing=nonfrench}

% Improve the `outline` environment by coloring it to make it stand out a bit
% (but don't have `\0` have any coloring).
\newenvironment{outlineitemizeenv}{\begin{itemize}\color{MidnightBlue}}{\end{itemize}}
\renewcommand*{\outlinei}{outlineitemizeenv}
\renewcommand*{\outlineii}{outlineitemizeenv}
\renewcommand*{\outlineiii}{outlineitemizeenv}
\renewcommand*{\outlineiiii}{outlineitemizeenv}

% Convenient RQ subsections, that can also be labeled and referred to via
% regular commands. If you want a regular list, rather than subsections for
% each, use the `researchquestions` environment defined below.
\newcounter{rqcounter}
\renewcommand*{\therqcounter}{RQ\arabic{rqcounter}}
\newcommand*{\rqsubsection}[1]{%
  \refstepcounter{rqcounter}%
  \medbreak%
  \noindent{}\enskip{}%
  \textit{\therqcounter{}: #1}%
  \smallskip%
}

% Convenient research-questions environment for defining RQs as a list. If you
% want subsections instead of a list, use the `\rqsubsection` command instead.
\newlist{researchquestions}{enumerate}{1}
\setlist[researchquestions]{label={\bfseries RQ\arabic*.},ref={\bfseries \arabic*},labelindent=\parindent,leftmargin=*}
\crefname{researchquestionsi}{\bfseries RQ}{\bfseries RQs}

% Make listings nicer by default
\lstset{
  % Use the straight single quote, rather than the bent one
  upquote=true,
  % Allow backticks to add latex in the middle of listing
  escapechar=`,
  % Make each listing narrow-monospaced, and use small text
  basicstyle=\footnotesize\ttfamily,
  % Improve look of spacing
  columns=spaceflexible,
  % Don't drop spaces to fix column alignment
  keepspaces=true,
  % Show spaces inside strings
  showstringspaces=true,
  % But don't show spaces or tabs in general
  showspaces=false,
  % Nor show tabs
  showtabs=false,
  % Comments should be grayed out
  commentstyle=\color[gray]{0.4},
  % Place a nicer box around the code with a gray shadow (use `single` instead
  % to just have a single line)
  frame=shadowbox,
  rulesepcolor=\color{gray},
  % Automatically un-indent code, so that we don't have to have code in the
  % LaTeX touch the left edge of the margin. Unfortunately lstautodedent is not
  % on CTAN, so needs to be got from https://github.com/jubobs/lstautodedent
  autodedent,
}
% Add the ability to get consistent line numbers in listings simply by adding a
% `style=WithLineNos`
\lstdefinestyle{WithLineNos}{
  numbers=left,
  firstnumber=auto,
  numberblanklines=true,
  numberstyle=\color{gray},
  numbersep=5pt,
  xleftmargin=15pt,
}

% Make it easier to define a short caption that repeats its prefix in the list
% of figures, but has extra text in the main body.
\newcommand*{\shortcaption}[2]{\caption[#1]{#1\@. \textit{#2}}}

% Switch to alternate font only for the typewriter font
\renewcommand*{\ttdefault}{cmtt}

% Make epigraphs be at a reasonable width
\setlength\epigraphwidth{.6\textwidth}

% Make `\_` "smarter" by recognizing when it is in a tt context, thereby setting
% itself to `\char95` (equivalent to `\string_`) which looks nicer
%
% Depending on use case, it might be helpful for this to be either defined via
% `\iftt` or `\ifttdefault`, thus both of them are included here.
%
% References: https://tex.stackexchange.com/a/643285 and https://tex.stackexchange.com/a/511530
\makeatletter
\newcommand\ifttdefault{%
  \edef\@tempa{\f@family}%
  \edef\@tempb{\ttdefault}%
  \ifx\@tempa\@tempb%
    \expandafter\@firstoftwo%
  \else
    \expandafter\@secondoftwo%
  \fi
}
\newcommand\iftt{%
  \begingroup
  \setbox0=\hbox{.}%
  \setbox1=\hbox{M}%
  \ifdim\wd0=\wd1
    \expandafter\endgroup\expandafter\@firstoftwo%
  \else
    \expandafter\endgroup\expandafter\@secondoftwo%
  \fi
}
\makeatother
\def\_{\iftt{\char95}{\textunderscore}}

% Important commands I think are always good to have, independent of paper
\newcommand*{\naive}[0]{na\"{\i}ve\xspace}
\newcommand*{\Naive}[0]{Na\"{\i}ve\xspace}
\newcommand*{\naively}[0]{na\"{\i}vely\xspace}
